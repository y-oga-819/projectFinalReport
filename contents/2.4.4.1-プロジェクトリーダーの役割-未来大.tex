\subsubsection{公立はこだて未来大学}
\par
ミライケータイプロジェクトのプロジェクトリーダの大きな役割の1つは,
連携を取っている3大学を取りまとめる全体のプロジェクトリーダとして,プロジェクト全体の計画を立て,
3大学が円滑に連携を行うことが出来るようスケジュール管理やタスク管理を行うことである.
具体的には,常にプロジェクト全体を見通して,次に何をしなければならないかを把握し,スケジュールを立てて提示し,
各活動ごとにリーダを立ててメンバにタスクを割り振り,メンバそれぞれの進捗を毎週のプロジェクト活動時間や
合同会議の場でヒアリングを行って把握することで,スケジュールを改めて立てなおしたりといった方法で,
プロジェクトが滞り無く進行するように調整を図ることが未来大のプロジェクトリーダの役割である.
また,毎週水曜日に行う合同会議や,各大学のプロジェクトリーダがSkypeチャット上で行うリーダ会議においては
ファシリテータを担当し,議題を考えたり,司会進行を行ったりといったことを指揮した.

未来大のプロジェクトリーダとしての役割は,未来大プロジェクトメンバのタスク管理やスケジュール管理を行うことである.
また,活動時にはファシリテータを担当し,その日に行う議題や次の活動までの課題などを提示して,
プロジェクトメンバがそれぞれ「自分がいま何を担当していて,いつまでに何をしなければならないのか」を
認識できているような状態にすることが重要な役割といえる.

\bunseki{小笠原 佑樹(未来大)}
