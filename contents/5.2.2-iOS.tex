\subsection{iOS班}
\par この節では,本プロジェクトのアプリケーションである「Rhyth/Walk」のiOS版を開発するために行ったプロセスについて述べる.
\par 前期
\begin{itemize}
\item iOSアプリケーション開発のためのXcodeの環境構築
\item Objective-C言語とSwift言語のプログラミングおよび開発工程を効率よく学ぶため,インターネットでキーワード検索をし,簡単なアプリケーションを作成.
\item 画面遷移図の完成
\item Objective-C言語とSwift言語のプログラミングの基本を勉強
\item Xcodeの基本操作の勉強
\item 使用言語をSwiftに決定

\par 後期のプロセスを述べる
\item 開発スケジュールの決定
\item 決定した画面遷移図を元に画面の開発
\item 各機能のアルゴリズムを作成
\item 全ての機能の画面をマージ
\item データの内部保存の実現
\item 未実装機能の実装スケジュール(優先度)を決定
\item 未実装機能の実装
 \end{itemize}
\bunseki{村上 惇(未来大)}
