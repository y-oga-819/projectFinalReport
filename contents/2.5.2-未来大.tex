\subsection{公立はこだて未来大学}

\par
未来大の活動の流れは,アイディアの提案・決定,要求定義・要件定義の実施,アプリケーショ
ン設計のためのサービス仕様書・詳細仕様書の作成,中間報告会での発表という流れで行った.
なお,開発に必要な技術の習得とビジネスモデルの考案は,これらの活動と平行して行った.

\par
前期は,企画の行程として,アプリケーションのアイディア出し・ブラッシュアップを行い,第一回合同合宿で
神奈工と長崎大とのアイディア交流を行い,本プロジェクトで開発するアプリケーションの決定を行った.
また,設計の行程として,要求定義書・要件定義書・サービス仕様書・詳細仕様書の作成を行い,
アプリケーションの詳細を話しあって認識の共通化をはかり,後期から開始する開発に備えた.
また,プロジェクト学習の成果報告の一環として,中間報告会でポスターとスライドを使用して活動状況の途中経過を発表した.

\par
後期は,開発の行程として,3大学が連携をとって実際にアプリケーションの開発を行った.
未来大では,「Cool Jpanimation」をAndroidとHTML5で,「Rhyth/Walk」をAndroidとiOSで開発することにし,
メンバがそれぞれのグループに分かれて開発を進めた.

また,ビジネスモデルの考案の行程として,未来大のメンバでそれぞれのアプリに対して活用シーンを考え,利益関係を考えた.

\par
以下は,プロジェクト学習の月ごとの年間スケジュールである.ここでは記載していないが,プロジェクト学習の
活動時間においてアイディア出し・ブラッシュアップ,技術習得の進捗確認を行っている.
\par

4月
\begin{itemize}
\item プロジェクト発足
\item プロジェクトメンバ顔合わせ
\item 神奈工との合同ミーティング,顔合わせ
\item プロジェクトリーダ他各種役割の分担
\item アイディア提案のグループを3グループに分担
\item 技術習得の班をAndroid,iOS,HTML5,サーバの4グループに分担
\item 未来大プロジェクトメンバは20個のアプリケーションアイディアの考案
\end{itemize}
5月
\begin{itemize}
\item 長崎大がプロジェクトに参加し,合同ミーティングで顔合せ
\item アイディア班の中でアイディアを6つに絞り,アイデア提案シートに記入
\item アイディア班の中でアイディアを4つに絞り,アイディアをブラッシュアップ
\item アイディア班の中でアイディアを2つに絞り,アイディアをブラッシュアップ
\item 各アイディア班の2つのアプリケーションについて発表会
\item 発表会の結果を元に,合宿に持っていく4つのアプリを決定
\item アイディア版を再編成し,4つの班を作成してアプリアイディアのブラッシュアップ
\item 各アイディアについて類似アプリケーションの調査
\item 合宿に持っていく4つのアプリに関して,ビジネスモデルの考案
\item 合宿にむけたデモ開発のために技術習得
\item IDY本田様の講演
\item 専修大学 渥美幸雄様の講演
\end{itemize}
6月
\begin{itemize}
\item 第一回合同合宿
\item 要求定義書の作成
\item 要件定義書の作成
\item サービス仕様書の作成
\item 詳細仕様書の作成
\item 中間報告書の作成に向けて,章立ての作成
\item 中間報告書の作成
\item 個人報告書の作成
\item 中間発表会のスライドを作成
\item 中間発表会のポスターを作成
\item サーバ班がOSSセミナーに出席
\item 中間発表会のポスター印刷のため,グラフィック工房の使用方法を学ぶセミナーに参加
\end{itemize}
7月
\begin{itemize}
\item サービス仕様書の作成
\item 詳細仕様書の作成
\item 中間報告書の作成
\item 個人報告書の作成
\item 中間発表会のスライドを作作成
\item 中間発表会のポスターを作成
\item TeXの講習会に参加
\item オープンキャンパスの準備
\end{itemize}
8月
\begin{itemize}
\item オープンキャンパスに参加
\item 実装のための技術習得に参加
\end{itemize}
9月
\begin{itemize}
\item アプリケーションの開発,機能を実装
\end{itemize}
10月
\begin{itemize}
\item アプリケーションの開発,機能を実装
\item アプリケーションのビジネスモデルを考案
\item キャンパスベンチャーグランプリへビジネスモデルを提出
\end{itemize}
11月
\begin{itemize}
\item アプリケーションの開発,機能を実装
\item 第二回合同合宿のためのスライドを作成
\item 第二回合同合宿
\item 最終報告会のスライドを作成
\item 最終報告会のポスターを作成
\end{itemize}
12月
\begin{itemize}
\item アプリケーションの開発,機能を実装
\item プロジェクトの最終報告会
\item 最終報告書の作成
\end{itemize}
1月
\begin{itemize}
\item 最終報告書を作成
\item 企業報告会の準備
\end{itemize}
2月
\begin{itemize}
\item 企業報告会の準備
\item 企業報告会
\end{itemize}
\bunseki{小笠原 佑樹(未来大)}
