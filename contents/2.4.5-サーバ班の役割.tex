\subsection{サーバ班の役割}
\par 未来大のサーバ班はリーダが1名,班員が2名の計3名で構成されており,長崎大のサーバ班はリーダが1名,班員が1名の計2名で構成されている.
以下にサーバ班の活動内容について述べる.
\\
\par <Wikiサーバの構築>
\par
本プロジェクト内における情報や,議事録や活動記録などのファイルの共有,スケジュール管理はWikiを用いて行った.
例年,本プロジェクトではプロジェクト開始時にWikiを動かすためのサーバ構築を行っており,今年もサーバ班では
Wikiサーバの構築を行った.本年度は以下の構成でWikiサーバを構築した.
\\
\par ・IPアドレス:210.226.0.74
\par 教員から借りたIPアドレス
\par ・OS:CentOS5
\par WikiサーバのOS
\par ・仮想:VMware vSphere Client
\par OSを仮想上で稼働させるためのもの
\par ・Web:Apache(httpd) 2.1
\par Webサーバを稼働させるWebエンジン
\par ・PHP:5.4
\par Wikiを動作させるためのもの
\par ・Wiki: PukiWiki 1.4.7
\par 3大学情報共有のためのWiki
\\
\par
利便性を確保するため,グローバルIPアドレスを1つ教員から借りた.無料で登録可能なDNSに登録し,
このWikiサーバをインターネット上に公開した.ただし機密性確保のため,Wikiの閲覧・編集は
パスワードにより保護を行っている.IPアドレスは1つしか無いが,ドメイン名でレスポンスを返す
サーバを振り分け,複数台のサーバがネットワーク外からアクセスできるように構築した.
\par
また,Wikiサーバをインターネットに公開した以上,その保守・管理も重要となる.セキュリティ確保
のため,以下のソフトウェアを導入した.
\\
\par ・ウイルス対策:Clam AntiVirus
\par ウイルスを検知・削除する
\par ・不正侵入対策:DenyHosts 2.6
\par SSHでの不正侵入を検知しIPアドレスを元にアクセスの権限を決定する.
\par ・改ざん防止:Tripwire 2
\par ファイルの変更を検知する.
\\
\par
管理はSSHによるリモートログインで行うが,セキュリティを考慮しパスワード認証でのログインを禁止した
上で,RSA鍵での認証を行っている.また,データディレクトリの自動バックアップを行うシェルスクリプト
を作成し,Wiki内のデータ保全を図っている.これらの定期的なログチェックを行い,不具合発覚時には,
バックアップの復元やサーバの調整など適切な対策を行った.
\\
\par <基本知識の習得>
\par
サーバについての基本的な知識の獲得を行った.主な習得方法としてはOSSセミナーの受講による知識獲得であった.
具体的にはLAMPを用いてWebサーバを動かすことやサーバのセキュリティ対策を行うこと,Webアプリケーション開発
を行うこと,jQueryによる動的なWebサイトの作成を行った.またサーバ通信についての知識獲得は,主にインター
ネットを利用することによって方法やプログラムの例を調査し,実際に通信テストを行うなどして知識の獲得を行った.
\\
\par <各アプリケーションサーバの構築>
\par
本プロジェクトで開発することとなった2つのアプリケーション「Cool Japanimation」と「Rhyth/Walk」の
アプリケーションサーバの構築を行った.各アプリケーションのサーバ班の役割については「Cool Japanimation」
は5.1.3に「Rhyth/Walk」は5.2.3にそれぞれ記載する.
\bunseki{鍋田 志木(未来大)}
