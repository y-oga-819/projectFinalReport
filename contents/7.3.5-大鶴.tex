\subsection{大鶴 宗慶}
5月
\begin{itemize}
\item 創生プロジェクトとしてこのプロジェクトに参加
\item 外国人に対してのアンケート調査
	長崎にいる留学生に以下の質問に対してアンケート調査を行った.
	\item 日本に来る前の日本のイメージ
	\item 実際に日本に来てイメージは変わったか
	\item 外国人にとってCool Japanimationというアプリはどうなのか
\end{itemize}
\par
6月
\begin{itemize}
\item CJ HTML5班メンバとして活動
	HTML5習得を開始した.
\item Cool Japanimationの要求定義書の議論
	チャット機能の担当に決定
\item 要求定義書の作成
	決まった担当箇所の定義書を作成した.
\item Cool Japanimationの要件定義書の議論
	チャット機能の担当に決定
\item 要件定義書の作成
	決まった担当箇所の定義書を作成した.
\end{itemize}
\par
7月
\begin{itemize}
\item Cool Japanimationのサービス仕様書の議論
	ツアー作成機能の担当に決定
\item Cool Japanimationの中間報告書の議論
	アプリケーションCool Japanimationの概要・目的の担当に決定.
\item 中間報告書の作成
	担当箇所の報告書を作成した.
\item サービス仕様書の作成
	決まった担当箇所の画面遷移図を作成した
\item レビュー機能を担当
	HTML5によるアプリの開発開始
\end{itemize}
\par
8月
\begin{itemize}
\item HTML5によるCool Japanimationの開発
	レビュー機能の開発
\end{itemize}
\par
9月
\begin{itemize}
\item HTML5によるCool Japanimationの開発
	レビュー機能の開発
\end{itemize}
\par
10月
\begin{itemize}
\item ナビ機能を担当
   HTML5にて開発開始
\end{itemize}
\par
11月
\begin{itemize}
\item 第二回合宿の参加(Skypeにて)
	長崎大は函館に行くことができなかったため,Skypeでの参加になった.
\item 創生プロジェクトの発表の準備
	スライド
	ポスター
	デモ
	アプリケーション紹介の練習
\end{itemize}
\par
12月
\begin{itemize}
\item 創生プロジェクト(ものづくり・アイデア展)
	会場の設営
	デモを含めたアプリケーション紹介
	懇親会の参加
\item Cool Japanimationの最終報告書の議論
Cool Japanimationのアプリケーションの概要・目的,長崎大と未来大の連携,長崎大学の前期・後記成果を担当することになった.
\item 最終報告書の作成
	担当箇所(第一次案)を作成した.
\item 最終報告書第一次案の修正・添削
	学生内でレビューを行い,修正を加える予定.
\end{itemize}
\par
1月
\begin{itemize}
\item 最終報告書第一次案の修正・添削
	先生方から意見をもらい,修正を加え第二次案を作成する予定.
\end{itemize}
\par
2月
\begin{itemize}
\item 企業報告会
\end{itemize}
\par
活動内容
 5月に行ったアンケート調査では,アンケートをとれた人数は少なかったが,一つの国に偏らず複数の国の外国人から意見が聞くことができたのでよかった.しかし,欧米出身の外国人にもアンケートを取ることができず,アジア出身の外国人に集中してしまったことが悔やまれた.
開発では,HTML5班の一員としてプロジェクトに参加した.未来大の方とレビュー機能を担当し,HTML5の学習を始めた.しかし,未来大の方との学習の差が開いてしまい,レビュー機能のプログラムのほとんどを任せた.10月頃から,ナビ機能を担当した.ナビ機能の開発を進めていく中で,HTML5やJava scriptについての知識が深まったのではないかと考える.開発以外では,学内の「ものづくり・アイデア展」での発表では,ポスターの作成,デモ機を用いながらのアプリケーションについての紹介を行った.アイデア展では,周りの方から多くの意見をいただくことができた.
 反省として,定義書や仕様書の作成の際,スケジュールの確認が甘く,編集がぎりぎりになることが多かった.

\bunseki{大鶴 宗慶(長崎大)}
