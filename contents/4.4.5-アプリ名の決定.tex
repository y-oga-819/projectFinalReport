\subsection{アプリ名の決定}
ここでは,アプリケーション名「Rhyth/Walk」が決定されるまでの流れを説明する.
最初に,アプリケーション提案時に仮タイトルとして付けられていた仮タイトルとして付けられていた「Rhyth-Walk」をそのまま仮タイトルとして決定した.\par
その後,夏休み直前にプロジェクトメンバーからタイトル案を募集し,Googleフォームを使用しアンケートを作成し,集められたタイトルの案を選別した.\par
集計の結果,「Rhyth/Walk」が一位となったのだが,アプリケーションの仕様変更やビジネスモデルの変更などの案が上がり,アプリケーション名の決定は一時保留された.\par
第二回合宿時,アプリケーションのビジネスモデルが確定し,それに伴いアプリケーションの名前も「Rhyth/Walk」と決定した.「Rhyth」と「Walk」の間が「-」から「/」に変わった理由は,「/」と「W」を併せることにより「Rhyth」と「Walk」の間に「M」があるように見せ,「Rhyth(M)/Walk」としての意味を含ませるためだ.\par
\bunseki{赤木 詠滋(神奈工)}
