\subsection{アプリ名の決定}
ここでは,アプリケーション名「Rhyth/Walk」が決定されるまでの流れを説明する.
最初に,アプリケーション提案時に仮タイトルとして付けられていた「Rhyth-Walk」をそのまま仮タイトルとして決定した.\par
その後,夏休み直前にプロジェクトメンバーからタイトル案を募集し,Googleフォームを使用しアンケートを作成し,集められたタイトルの案を選別した.\par
集計の結果,「Rhyth/Walk」が一位となったのだが,アプリケーションの仕様変更やビジネスモデルの変更などの案が上がり,アプリケーション名の決定は一時保留された.\par
第二回合宿時,アプリケーションのビジネスモデルが確定し,それに伴いアプリケーションの名前も「Rhyth/Walk」と決定した.「Rhyth」と「Walk」の間が「-」から「/」に変わった理由は,「/」と「W」を併せることにより「Rhyth」と「Walk」の間に「M」があるように見せ,「Rhyth(M)/Walk」としての意味を含ませるためだ.\par
何故「リズ(ム)ウォーク」なのか,その理由はこのアプリケーションが「音楽再生アプリケーションであること」と「移動しながら音楽を聴く人」を対象としているからだ.\par
「Rhyth」は音楽アプリケーションであることを,「Walk」は移動しながらということを,それぞれ意味し表している.
\bunseki{赤木 詠滋(神奈工)}
