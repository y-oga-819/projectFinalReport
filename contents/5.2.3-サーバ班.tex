\subsection{サーバ班}
\par
ここでは,「Rhyth/Walk」サーバ班の活動内容について述べる.以下に活動内容の詳細を示す.
\par
・技術習得
\par
サーバ開発の技術習得では前期はサーバを開発するにあたって必要な技術を身につけることに重点を置いた.具体的にはOSSセミナの受講による基本的な知識の獲得や,教員や昨年のミライケータイプロジェクトのサーバ班を担当した先輩に,今のサーバ班はどの様な動きをすべきなのかを相談するなどをした.後期では,Rhyth/Walkで行われるサーバ通信の処理フローの考案や処理に必要なPHPファイルの考案を行い,それに伴って必要となる技術の習得をするためにPHPやMySQLの学習を行った.
\par
・アプリケーションサーバの実装
\par
アプリケーションサーバの実装をするに当たって,サーバ班では開発した各プラットフォームのアプリケーションとの通信方法の知識が不足していたので,通信方法の調査を行ったのち実際にAndroid班の端末で通信テストを行った.またアプリケーションを基に必要な機能の洗い出しを行った.データベースについては,サーバ班だけでなくRhyth/Walkを開発しているメンバと話し合うことでどの様なデータベースが必要かや,それぞれのデータベースにどの様なデータを格納するのかなどを決定することにした.この様なRhyth/Walk全体で話し合うことが必要な事は主に毎週日曜日に行われているSkypeによる会議で決定している.
\par
\bunseki{鍋田 志木(未来大)}
