\subsection{ロゴの決定}
\par
私たちは,ロゴを決めるのにあたって,未来大で2つロゴの案を出して,ペイントやイラストレーションを使用し作成した.
長崎大は一人一つずつロゴの案を出し,手書きで書かれたものと,ペイントを使いロゴを作成した.
作成期間は12月3日から12月9日までの一週間であった.
提案されたロゴは,Googleドライブにあげ,プロジェクトメンバ全員が閲覧できる状態にし,12月10日(水)に各自が良いと
思ったロゴに投票して,一番投票数が多いものをCool Japanimationのロゴの原案とした.
投票形式は,WikiにCool Japanimationのロゴ用のページを用意し,提案されたロゴに番号をつけて,良いと思ったロゴの番号の投票ボタンをクリックするものとした.
ロゴは,これで最終決定したのではなく,選ばれたロゴをこれから少しずつ改良する方針にした.
選ばれたロゴは,富士山に,習字タッチで書かれたCool Japanimationの文字で,日本の日の丸をモチーフにした赤が背景の色として使われた.
しかし,このデザインのままであると,日本らしさと,旅行に行くニュアンスのアイデアが盛り込まれてはいるが,肝心のアニメらしさを上手く表現出来ていないものであった.
選ばれなかったロゴに関しても同じことが言えた.
そこで,私たちは,選ばれた原案をもとに改良が必要であると判断した.
これからの予定としては,今の原案のロゴを描き方を変更して,アニメらしさを表現する方法として,キャラクターを考えて付け足し,アニメ感を出そうかと考えている.
\bunseki{金澤 しほり(未来大)}
