\subsection{藤原 由美恵}
5月
\begin{itemize}
\item HTML5班メンバとして活動
\par 第一回合同合宿に持っていくHTML5のデモ開発を行うことになった.
\item 第一回合同合宿に向けた技術習得
\par GPSを使ったデモアプリケーションの作成を行った.
\item アプリケーションのアイディアを20個考案
\item 未来大内アプリケーションを3つ選定
\item 合同合宿で発表するアプリケーション「しゃべるんです」のアイディア提案シート・スライド作成
\end{itemize}
6月
\begin{itemize}
\item 第一回合同合宿に参加
\par 本プロジェクトで提案するアプリケーションの決定,技術習得の際に作成したデモアプリケーションを合同合宿で発表を行った.
\item アプリケーション「Cool Japanimation」 HTML5班に所属
\par 「Cool Japanimation」の開発を担当することになった.
\item 要求・要件定義書のリーダに就任
\item 「Cool Japanimation」についての話し合いの会議に参加
\item 要求定義書,要件定義書,サービス仕様書の作成
\par 長崎大学とSkype会議で議論しながら,要求定義書,要件定義書,サービス仕様書を作成した.
\end{itemize}
7月
\begin{itemize}
\item 詳細仕様書の作成
\par 項目ごとに担当を割り振り,長崎大とSkype会議で議論しながら詳細仕様書を作成した.
\item 「Cool Japanimation」についての話し合いの会議に参加
\item 中間発表の「Cool Japanimation」のサブスライドリーダーに就任
\item 中間発表スライド,ポスタの作成・使用する画像の用意
\par 「Cool Japanimation」のポスタ作成,掲載する画像を作成した.
\item HTML5 技術習得
\item 中間発表
\par 製作したポスタ,スライド,デモ機を用いて,アプリケーション「Cool Japanimation」の発表を行った.
\item 中間報告書の作成
\end{itemize}
8月
\begin{itemize}
\item アプリケーション開発のための技術習得
\par HTML5での開発に必要な技術を習得した.
\item 担当割り振り
\par 誰がどの機能を開発するか割り振りをした.
\item アプリケーションの開発
\par HTML5でのアプリケーション開発を進めた.
\end{itemize}
9月
\begin{itemize}
\item アプリケーションの開発
\par HTML5でのアプリケーション開発を進めた.
\end{itemize}
10月
\begin{itemize}
\item アプリケーションの開発
\par HTML5でのアプリケーション開発を進めた.
\item キャンパスベンチャーグランプリに応募
\par キャンパスベンチャーグランプリに「Cool Japanimation」を応募するため,背景・着想点・サービス内容について考案し,書類を作成した.
\par 評価会を行い,書類の修正を行った.
\item アカデミックリンクで公開するデモの作成
\par 「Cool Japanimation」のデモで使う機能を作成した.
\end{itemize}
11月
\begin{itemize}
\item アプリケーションの開発
\par HTML5でのアプリケーション開発を進めた.
\item アカデミックリンクに参加
\item 第二回合同合宿に向けた資料作成
\par メンバと話し合いながら,資料作成を行った.
\item 第二回合同合宿に参加
\item 最終発表に向けたスライドの作成
\par 「Cool Japanimation」についてのスライドを作成した.また,発表練習を行い,評価をもとに修正を行った.
\end{itemize}
12月
\begin{itemize}
\item アプリケーションの開発
\par HTML5でのアプリケーション開発を進めた.
\item 最終発表に向けたスライドの作成
\par 「Cool Japanimation」についてのスライドを作成した.また,発表練習を行い,評価をもとに修正を行った.
\item 最終発表
\par ポスタ,スライド,デモを用いて「Cool Japanimation」の発表を行った.
\item 最終報告書の作成
\end{itemize}
1月
\begin{itemize}
\item アプリケーションの開発
\par HTML5でのアプリケーション開発を進める.
\item 最終報告書の作成
\item 企業発表用のスライド作成
\par 他校メンバ,リーダーと協力して作成する.
\item ドキュメント再編集
\par これまでに作成したドキュメントの再編集を行う.
\end{itemize}
2月
\begin{itemize}
\item 秋葉原課外発表会
\par 秋葉原課外発表会に参加する.
\item 企業報告会
\par 本プロジェクトの最終目標である企業報告会に参加する.
\end{itemize}

活動内容と予定
\par 本プロジェクトにおいて,私はHTML5班に所属し,「Cool Japanimation」に関するものを担当した.
ドキュメントを作成する際には,メンバと何度も納得のいくまで繰り返し修正や改善を行った.他校のメンバにもこまめに連絡を取り合い,メンバ間で情報や意見を何度も交し合いドキュメント作成を行った.開発に関して,スケジュール通り進めることができず遅れてしまうことがあった.その際は進捗確認で遅れていることを共有して対処するよう努力した.
\par また,プロジェクト活動では活動の記録を見やすくまとめれるように心がけた.板書では,重要なところに色を付けたりと見易さに気をつけて情報が伝わるようにした.板書以外のときは板書のときよりも書き漏らしが無いよう心がけた.日ごろの活動や発表練習で先生方先輩方から頂いたコメントはいつもしっかりを書き留め,反映するためにどのようにすべきか考えながら活動を行った.
\par これからは,本プロジェクトの最終ゴールである企業報告会の準備に向けてスライド,ポスタの作成,開発,ドキュメントの編集などをしていく予定である.

\bunseki{藤原 由美恵(未来大)}
