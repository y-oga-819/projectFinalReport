\subsection{赤木 詠滋}
アプリケーションの開発ではAndroid版「Rhyth/Walk」を担当しており,個人の作業として,主に必要書類の作成や,音楽解析のスケール検出を担当していた.スケール解析では,波形データの周波数成分をFFTを用いて検出し,そのデータを用いて音階を判定するための条件式の構築などを行っていた.
4月
\begin{itemize}
\item セミナーに参加
\end{itemize}
5 月
\begin{itemize}
\item Androidアプリケーション製作のための技術習得
\item アプリケーションのアイディア出し
\item アイディアシートの作成
\item 第一回合同合宿の資料作成のフォロー
\item 技術習得報告デモの作成
\end{itemize}
6 月
\begin{itemize}
\item 第一回合同合宿
\item 合宿グループワークでのビジネスモデル係
\item 要求定義書リーダ会議
\item 要件定義書リーダ会議
\item 要件定義書の作成
\item 類似アプリケーション調査
\end{itemize}
7 月
\begin{itemize}
\item サービス仕様書のための会議に参加
\item 中間報告書の作成
\item 中間発表
\end{itemize}
8 月
\begin{itemize}
\item 必要な技術を習得
\item アプリケーション開発
\end{itemize}
9 月
\begin{itemize}
\item 必要な技術を習得
\item アプリケーション開発
\end{itemize}
10 月
\begin{itemize}
\item 必要な技術を習得
\item アプリケーション開発
\end{itemize}
11 月
\begin{itemize}
\item 第二回合同合宿
\item アプリケーション開発
\end{itemize}
12 月
\begin{itemize}
\item 学内最終発表準備
\item 学内最終発表
\item プロジェクト最終報告
\item 最終報告書の作成
\end{itemize}
1 月
\begin{itemize}
\item 最終報告書の作成
\item 最終報告会
\end{itemize}
2 月
\begin{itemize}
\item 企業報告会の準備
\item 企業報告会
\end{itemize}
個人作業として特筆すべき点は,一番最初に書いたが「スケール解析」の部分が挙げられる.現段階ではまだ完成には程遠いとはいえ,ピアノソロの曲ではかなり正確なスケールの検出が可能なレベルまで持って来れている.また,まだ検討段階ではあるが,私が担当しているスケール解析ではそれを元にして,人の感情との関連付けを行えないかという事を考えている.これは楽曲の波形データを取り扱う部門でもまだまだ研究段階の物であり,私が行っていた事は卒業研究とほぼ同定義であったらしい.
\bunseki{赤木 詠滋(神奈工)}
