\subsection{小笠原 佑樹}

以下にミライケータイプロジェクトの個人活動を示す.

プロジェクトリーダに就任し,常に全体を見通してこれからのスケジュールとやらなければならないタスクを提示して,
メンバが作業にとりかかるための提起を行っていた.さらに,第二回合同合宿の前までは企画の段階に参加してアプリケーションの
アイディアを考案した.企画以降の設計,開発,ビジネスモデルの作成については,直接アプリケーション担当メンバとして
関わるのではなく,プロジェクトリーダとしてプロジェクト全体を俯瞰しながら,それぞれの活動に対してレビューを行う
という形で関わった.開発に関しては,リソースの足りていない部分に関わって,実際に開発を行った.
また,学内での発表会などの機会には,プロジェクト全体の概要を説明する部分の発表スライド作成を担当した.


4月
\begin{itemize}
\item プロジェクトリーダに就任した
\item アプリケーション案を20個考える
\item アイディア班Cグループに参加した
\item 技術習得のiOS班に参加した
\end{itemize}
5月
\begin{itemize}
\item 合同ミーティングのファシリテータを担当した
\item アイディア班の中でアプリアイディアを6つに絞り,アイディア提案シートに記入した
\item アイディア班の中でアプリアイディアを4つに絞り,アイディアをブラッシュアップした
\item アイディア班の中でアプリアイディアを2つに絞り,アイディアをブラッシュアップした
\item 各アイディア班の2つのアプリについて発表会を開いた
\item 発表会の結果を元に,合宿に持っていく4つのアプリを決定した
\item アイディア版を再編成した
\item 4つの班を作成してアプリアイディアのブラッシュアップをした
\item 各アイディアについて類似アプリケーションの調査を行った
\item アイディアのビジネスモデルの考案をした
\item 合宿にむけたデモ開発のためにObjective-Cの技術習得を行った
\end{itemize}
6月
\begin{itemize}
\item 合同ミーティングのファシリテータを担当した
\item 第一回合同合宿に参加した
\item 中間報告書の作成に向けて担当を割り振った
\item 中間報告書の作成をした
\item 個人報告書の作成をした
\item 中間発表会のプロジェクト全体に関するスライドを作成した
\item オープンキャンパスの準備をした
\end{itemize}
8月
\begin{itemize}
\item オープンキャンパスに参加した
\item 実装のための技術習得に参加した
\end{itemize}
\newpage
9月
\begin{itemize}
\item アプリケーションの開発,機能を実装した
\end{itemize}
10月
\begin{itemize}
\item 合同ミーティングのファシリテータを担当した
\item Cool Japanimationのビジネスモデルにレビューを行った
\item Rhyth/Walkのビジネスモデルにレビューを行った
\item Cool Japanimationのレビュー機能を実装した
\item Rhyth/Walkの歌詞解析アルゴリズムの一部を提案した
\item Android端末とアプリケーションサーバとの通信を実装した
\item サーバ通信確認用のデスクトップアプリケーションを開発した
\item サーバ側でデータベースに接続するスクリプトを書いた
\item 第二回合同合宿で使用するプロジェクト全体の進捗確認用スライドを作成した
\end{itemize}
11月
\begin{itemize}
\item 合同ミーティングのファシリテータを担当した
\item 第二回合同合宿に参加した
\item 最終報告会で使用するプロジェクト全体の説明スライドを作成した
\end{itemize}
12月
\begin{itemize}
\item 合同ミーティングのファシリテータを担当した
\item プロジェクトの最終報告会に参加した
\item 最終報告書の作成を行った
\item 最終報告書のレビューを行った
\item 学習フィードバックシートの作成を行った
\item 個人報告書の作成を行った
\item プロジェクト報告書の作成を行った
\item 企業報告会の準備を行った
\end{itemize}
1月
\begin{itemize}
\item 最終報告書の作成をする
\item 最終報告書のレビューを行う
\item 企業報告会の準備を行う
\end{itemize}
2月
\begin{itemize}
\item 企業報告会の準備を行う
\item 企業報告会を行う
\end{itemize}

全体を通して,プロジェクトリーダとして常に全体を把握しつつ,最も重要度の高い作業から順番に効率よく進めていくために
スケジュールを調整することがこの1年間を通しての主な作業であった.ドキュメントの作成やアプリケーションの開発,
学内での報告会等がある度にリーダを立てて,はじめにガントチャートを用いたスケジュールを作成するように徹底することで,
視覚的にメンバがスケジュールを把握できるように務めた.遅れが出る度にスケジュールを再調整してガントチャートを提示し,
現在の状況を知ることが出来るように気をつけていた.

また,後期はアプリケーションの開発をするにあたって,他のメンバが忙しくて手が付けられていない機能などを引き受けて,
実際に開発を行うことで開発が滞らないようにした.具体的には,Android班のサーバ連携部分の処理が,技術調査の段階で
開発が滞っており,作業リソースと開発の優先順位の問題から放置されていたので,担当を引き受けてJSON形式での
サーバ通信アルゴリズムを作成し,メンバが使えるように共有した.

\bunseki{小笠原 佑樹(未来大)}
