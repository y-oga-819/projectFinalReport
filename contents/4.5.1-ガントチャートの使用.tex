\subsection{ガントチャートの使用}
\par
私たちは,プロジェクトを進めていくのにあたり,様々なイベントや作業ごとにスケジュール立てを必ず行った.
その際に,今年は,スケジュールが一目でわかるようにガントチャートを作成した.
ガントチャートを作成するようになったのは,プロジェクトの初回時に,プロジェクトリーダーがガントチャートを作成して,
スケジュールを説明していたので,後のプロジェクト活動に,プロジェクトメンバーがガントチャートを用いるようになったためであった.
企業などでは,ガントチャートがよく用いられていることから,作成の仕方を学ぶことができた.
また,ガントチャートを作成する際に,絶対的な締め切りを意味するマイルストーンを常に意識することができた.
このことより,スケジュールを間に合わせるためには今何をしなければならないかということを,
スケジュールを逆算して考えて,プロジェクト活動を行うことができた.
\bunseki{金澤 しほり(未来大)}
