\subsection{澤田 隼}
\par
本プロジェクトにおいて,Androidでの開発を行い,主に「Rhyth/Walk」を担当した.
開発するアプリケーションのアイディアとして「Rhyth/Walk」の原案を考案したこともあり,「Rhyth/Walk」のリーダを務めた.
\par
前期は,主に仕様書の作成を行った.
まず,合宿に向けてアイディア出しを行い,メンバと何度も話し合いを重ね,アイディアのブラッシュアップを行った.
そして,合宿でこのアイディアを発表し,このアイディアを採用してもらえるように,何度もスライドを修正したり,プレゼンの練習を行った.
また,Androidでアプリケーションを開発するために必要な技術習得を行った.
合宿で「Rhyth/Walk」をつくることに決まった後は,「Rhyth/Walk」の仕様書の作成を行った.
自分の中に「Rhyth/Walk」のイメージがあったこともあり,積極的に関わった.
\par
後期は,主に「Rhyth/Walk」の開発を行った.
スマートフォンに内蔵されている加速度センサや,周波数解析の技術習得と並行して「Rhyth/Walk」を開発した.
「Rhyth/Walk」のリーダとしてAndroidとiOSの開発の進捗管理や,開発タスク管理表を作成し,「Rhyth/Walk」の神奈工との合同会議の進行を行った.

5月
\begin{itemize}
\item 合宿に向けたアイディア出し
\item Android技術習得に所属 
\item 第一回合同合宿用デモ作成(Android)
\end{itemize}
6月
\begin{itemize}
\item 合宿に向けたアイディアのブラッシュアップ
\item 合宿で発表するアイディアの発表スライドの作成
\item 第一回合同合宿
\item Android班「Rhyth/Walk」リーダに就任
\item 要求定義書の作成
\item 要件定義書の作成
\item 「Rhyth/Walk」アプリケーション名コンテスト開催
\end{itemize}
7月
\begin{itemize}
\item 要求定義書のための会議に参加
\item 要件定義書のための会議に参加
\item サービス仕様書のための会議に参加
\item 中間発表会用の「Rhyth/Walk」のポスターを作成 
\item 中間発表会用の「Rhyth/Walk」のスライド作成協力
\item 中間発表会第1次「Rhyth/Walk」の台本の作成協力
\item 中間発表会用デモ作成(Android)
\item 中間発表会
\end{itemize}
8月
\begin{itemize}
\item オープンキャンパスの用意
\item 必要な技術を習得
\item アプリケーション開発
\end{itemize}
9月
\begin{itemize}
\item 必要な技術を習得
\item アプリケーション開発
\item 「Rhyth/Walk」の定期合同会議
\end{itemize}
10月
\begin{itemize}
\item 必要な技術を習得
\item 開発タスク管理表の作成
\item アプリケーション開発
\item 「Rhyth/Walk」の定期合同会議
\end{itemize}
11月
\begin{itemize}
\item アカデミックリンク
\item アプリケーション開発
\item 合宿で使用する発表スライドの作成
\item 第二回合同合宿
\item 最終発表用の「Rhyth/Walk」のイメージ動画作成
\item 最終発表用の「Rhyth/Walk」のスライド作成協力
\item 「Rhyth/Walk」の定期合同会議
\end{itemize}
12月
\begin{itemize}
\item 最終発表用「Rhyth/Walk」のデモ作成(Android)
\item プロジェクト成果発表会
\item 最終報告書の作成
\item 最終報告書のレビュー
\item 「Rhyth/Walk」の定期合同会議
\end{itemize}
1月
\begin{itemize}
\item 企業報告会の準備
\item 最終報告書の作成
\item サービス仕様書の修正
\item 詳細仕様書の修正
\item アプリケーション開発
\end{itemize}
2月
\begin{itemize}
\item 企業報告会の準備
\item 企業報告会
\end{itemize}

\par
プロジェクトとしてアプリケーションを開発する際,スケジュール通りに開発が進まなかったり,情報共有がしっかりできていなかったなどさまざまな課題に直面した.
そこで開発までにやらなければならないものを網羅的かつ詳細に書き出した開発タスク表を作成したことは,大いに役立った.
共同で開発を行う上で依存関係を明確にすることや,開発タスクの視覚化と共有は大切であると学んだ.

\bunseki{澤田 隼(未来大)}
