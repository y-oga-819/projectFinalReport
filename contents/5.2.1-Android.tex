\subsection{Android班}

\par この節では,本プロジェクトのアプリケーションである「Rhyth/Walk」を開発するために行なったプロセスについて述べる.
\begin{itemize}
\item Androidアプリケーション開発のためのeclipceの環境構築
\item Androidプログラミングおよび開発工程を効率よく学ぶため,参考書のサンプルコードを参考に簡単なアプリを作成.
\item 画面遷移図の完成
\item Androidプログラミングの基本を勉強
\item eclipseの基本操作の勉強

\par 後期のプロセスを述べる
\item 開発スケジュールの決定
\item 決定した画面遷移図を元に画面の開発.
\item 画面を実際に遷移させる
\item 各機能のアルゴリズムを作成
\item 全ての機能の画面をマージ
\item データの内部保存の実現
\item 未実装機能の実装スケジュール(優先度)を決定
\item 未実装機能の実装
 \end{itemize}
 
\bunseki{中司 智朱希(未来大)}
