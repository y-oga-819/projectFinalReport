\subsection{Android班}
前期
\begin{enumerate}
\item 開発担当についての問題
\par
神奈工がAndroidでRhyth/Walkを実装し,長崎大がHTML5でCool Japanimationを実装することになった.また,未来大のiOSはRhyth/Walkを担当し,長崎大と同様未来大のHTML5はCool Japanimationを担当となった.未来大のAndroidが固まってどちらかのアプリを担当するとリスク分散などのバランスが悪くなるため,未来大のAndroidはRhyth/WalkもCool Japanimationも担当することになった.
\item サーバ班との連携通信方法についての問題
\par
サーバとの通信がどのように行うの話合い,サーバとの通信方法についてはhttpdを使用する方針になっている.
\item 中間報告会に向けたデモアプリ開発による問題
\par
仕様書や中間報告会のスライド・ポスター作成に力を注ぎ,進行が遅れてしまった.中間報告会では画面遷移ができるデモを用意し,完成度を低めにし中間報告会に間に合わせた.
\end{enumerate}

\par 後期
\begin{enumerate}
\item Androidのバージョンの認識のずれ
\par
メンバ間で話し合い,デモに使用する実機のバージョンも考え統一した.
\item eclipseの不具合
\par
メンバ間で不具合の原因,エラーメッセージなどを共有し,対処した.基本的にはeclipseの再インストールにより解決していた. 
\item 各機能のアルゴリズムの実装の遅れ
\par
歌詞解析,テンポ解析のアルゴリズムの実装がスケジュールよりも遅れてしまうことがあった.メンバ間で相談し合い,優先度をつけ最低限の目標を決め,実装することとなった.
\item 「Cool Japanimation」の同時開発による進捗の遅れ
\par
「Cool Japanimation」との同時開発のため,集中的に開発できず,進捗が遅れてしまうことが多々あった.そのため4人を2人ずつ分け、各担当の機能実装を委託し集中し効率的に開発することになった.
\item サーバ連携実装の遅れ
\par
画面の仕様や機能のアルゴリズムに固執してしまい,サーバ連携が滞ってしまった.合同合宿や最終発表会がありデモも見せるため,サーバ連携を後回しにし機能実装を優先的に進めている.
\bunseki{中司 智朱希(未来大)}
