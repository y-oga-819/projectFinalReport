\subsection{Android班}
前期
\begin{enumerate}
\item 開発担当についての問題
\par
未来大のAndroid班では班の中で開発するアプリケーションを決める際に希望を取ったところ,1つのアプリケーションに偏ってしまった.
メンバは4人なので半数に分ければよいが,長期の開発にあたりモチベーションが維持できるかが問題となり
「Rhyth/Walk」だけでなく「Cool Japanimation」も開発することとなった.
\item サーバ班との連携通信方法についての問題
\par
サーバとの通信がどのように行うかを話合い,サーバとの通信方法についてはhttpsを使用する方針になった.
\item 中間発表会に向けたデモアプリ開発による問題
\par
仕様書や中間発表会のスライド・ポスター作成に力を注ぎ時間を割いてしまい,開発に十分な時間が確保できなかった.
また,音楽解析の知識も不十分だったため,当日の使用したデモストレーションでは,画像を貼付けただけの画面遷移のみできる
実機を用意することになった.
\end{enumerate}

\par 後期
\begin{enumerate}
\item Androidのバージョンの認識のずれ
\par
Androidのバージョンの認識のずれがあり,歌詞解析で使うAPIがバージョン4.0以上でないと利用できないものであり,
当初設定していたバージョン2.3に合わせるのではなく,バージョン4.0を基準におくこととなった.開発にそういったツールを
使用する際にはバージョンに注意する事と,開発環境だけでなく実機のバージョンにも注意するようにメンバ間で意識統一を行った.
\item 開発ソフトEclipseの不具合
\par
開発ソフトEclipseで不具合がおこることが多く,メンバ間で同じエラーが起こり,各自で解決することがあった.
同じエラーにメンバ複数人が対処するは効率が悪いと判断し,不具合を見つけた場合,メンバ間で不具合の原因,
エラーメッセージなどを共有し,解決策も共有するようにした.
\item 各機能のアルゴリズムの実装の遅れ
\par
歌詞解析,テンポ解析のアルゴリズムの実装がスケジュールよりも遅れてしまうことがあった.優先度をつけ最低限の目標を決め,リスケジュールを行った.
\item 「Cool Japanimation」の同時開発による進捗の遅れ
\par
「Cool Japanimation」との同時開発のため,集中的に開発できず,進捗が遅れてしまうことが多々あった.そのためメンバ4人を2人ずつ分け、各担当の機能実装を委託し,集中することによって効率的に開発することになった.
\item サーバ連携実装の遅れ
\par
画面の仕様や機能のアルゴリズムに固執してしまい,サーバ連携が滞ってしまった.合同合宿や最終発表会がありデモも見せるため,サーバ連携を後回しにし機能実装を優先的に進めている.
\bunseki{中司 智朱希(未来大)}
