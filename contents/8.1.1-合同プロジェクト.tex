\subsection{合同プロジェクト}

\par 私たちは合同プロジェクトにおいて未来大,神奈工,長崎大の3校が合同で「Rhyth/Walk」,「Cool Japanimation」の2つのアプリケーションの提案・開発を行った.3校合同でアプリケーションのアイディア提案を行い,後にアプリケーションの開発を行っている.アイディアの提案からアプリケーションの開発の活動を詳細を以下に示す.
\par 3校がそれぞれアプリケーションについてのアイディアを出し合い,アイディアを絞った.その後,未来大が4つのグループ,神奈工が1つのグループ,長崎大が1つのグループに分かれて6月7日,6月8日の2日間にわたって行った第一回合宿に向けてプレゼンテーションの準備をした.まず未来大では,各グループのメンバが持ち寄ったアイディアをもとに各グループ2つ,計6つのアイディアに絞った.その6つの中から学内代表のアイディアとなるものを投票の末4つに絞った.その4つのアイディアでグループメンバをシャッフルした.このようにすることによって,他メンバからの意見を参考にすることができた.意見をもとにアイディアをブラッシュアップし,プレゼンテーション用の資料としてまとめた.第一回合同合宿の1日目は,各大学の各グループがプレゼンテーションを行った後,他グループメンバからの意見や,企業の方々・OB の方々の意見を得た後に,大学間の境をなくした混合グループを編成した.新たなグループではそれまでに提案されたアイディアを参考に,新たなアイディアとしてまとめなおし,プレゼンテーション用の資料を作成した.合同合宿の2日目には新アイディアのプレゼンテーションの後,全プロジェクトメンバと企業の方々がアイディアの投票を行った.その結果,ミライケータイプロジェクトでは「Rhyth/Walk」「Cool Japanimation」という2つのアプリケーションを設計・開発することに決定した.合同合宿ではプロジェクトに関わるメンバが1箇所に集まり意見を交わし,また企業の方々・OBの方々から多くの意見・お話を聞くことができ,非常に有意義な時間を過ごすことができた.毎週水曜日に行うSkypeを利用した3大学合同会議では,各大学の進捗状況を確認するとともに意見・意思の共有しながら活動している.プロジェクト用のWikiも活用し,各会議の議事録や進捗状況などを共有している.具体的な活動内容としてはアプリ開発だけでなく,仕様書の作成,未来大ではプロジェクトの中間発表も行う.
\par 前期を通して,合同合宿での意見交換や企業,OB・OG,教員の方々からの熱心な指導,アドバイスによりプロジェクトの目的に向けたアプリケーションの提案の方法,実践的な開発方法の手法について学び,各校が成果を上げることができた.私たちはこの成果を後期のプロジェクトで役立てたい.

\bunseki{藤原 由美恵(未来大)}