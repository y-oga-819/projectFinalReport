\subsection{合同プロジェクト}

\par 私たちは合同プロジェクトにおいて未来大,神奈工,長崎大の3校が合同で「Rhyth/Walk」「Cool
Japanimation」の2つのアプリケーションの提案・開発を行った.3校合同でアプリケーションのアイ
ディア提案を行い,後にアプリケーションの開発を行っている.アイディアの提案からアプリケーショ
ンの開発の活動を詳細を以下に示す.
\par 3校がそれぞれアプリケーションについてのアイディアを出し合い,アイディアを絞った.その後,未
来大が4つのグループ,神奈工が1つのグループ,長崎大が1つのグループに分かれて6月7日,6月8日の 2 日間にわたって行った第一回合宿に向けてプレゼンテーションの準備をした.ま
ず未来大では,各グループのメンバが持ち寄ったアイディアをもとに各グループ2つ,計6つの
アイディアに絞った.その6つの中から学内代表のアイディアとなるものを投票の末,4つに絞った.
その4つのアイディアでグループメンバをシャッフルした.このようにすることによって,他人からの
意見を参考にすることができた.意見をもとにアイディアをブラッシュアップし,プレゼンテーション用
の資料としてまとめた.第一回合同合宿の1日目は,各大学の各グループがプレゼンテーションを
行った後,他グループメンバからの意見や,企業の方々・OB ・OGの方々の意見を得た後に,大学間の
境をなくした混合グループを編成した.新たなグループではそれまでに提案されたアイディアを参
考に,新たなアイディアとしてまとめなおし,プレゼンテーション用資料を作成した.合同合宿の2日
目には新アイディアのプレゼンテーションの後,全プロジェクトメンバと企業の方々がアイディア
の投票を行った.その結果,ミライケータイプロジェクトでは「Rhyth/Walk」「Cool Japanimation」という
2つのアプリケーションを提案・開発することに決定した.合同合宿ではプロジェクトに関わるメン
バーが1箇所に集まり意見を交わし,また企業の方々・OB・OGの方々から多くの意見・お話を聞くこと
ができ,非常に有意義な時間を過ごすことができた.毎週水曜日に行うSkypeを利用した3大学合
同会議では,各大学の進捗状況を確認するとともに意見・意思の共有しながら活動している.プロ
ジェクト用のWikiも活用し,各会議の議事録や進捗状況などを共有している.具体的な活動内容と
してはアプリ開発だけでなく,各種仕様書の作成,未来大ではプロジェクトの中間発表も行う.前期を
通して,合同合宿での意見交換や企業,OB・OG,教員の方々からの熱心な指導,アドバイスによりプロ
ジェクトの目的に向けたアプリケーションの提案の方法,実践的な開発方法の手法について学び,
各校が成果を上げることができた.私たちはこの成果を後期のプロジェクトで役立てたい.
\bunseki{藤原 由美恵(未来大)}
