\subsection{公立はこだて未来大学}
\par
未来大プロジェクト学習は,ミライケータイプロジェクトというテーマのもと,神奈工,長崎大と共同でスマートフォンやタブレットなどのモバイル端末を対象にモバイルアプリケーションの企画,開発を行うものである.
\par
本プロジェクトは,民間企業
\begin{itemize}
\item 日本ヒューレット・パッカード株式会社
\item Y!mobile株式会社
\item 株式会社エヌ・ティ・ティ・ドコモ
\item ソフトバンクモバイル株式会社
\item 株式会社サイバー創研
\item KDDI 株式会社
\item 株式会社NTC
\item IDY 株式会社
\end{itemize}
から,アドバイスやモバイル端末の提供などの協力をいただいている.また,これらの企業にプロジェクトメンバと一緒に合宿に参加して頂き,アプリケーションアイディアの発案や,プロジェクトの進行についてアドバイスをもらっている.
\par
また,プロジェクトの進行については,先輩からもアドバイスをもらい,相談した上でプロジェクトを進行する.
\par
本プロジェクトの目的は,モバイル端末の各種センサを活用し,既存のスマートフォンの枠組みにとらわれない新しい発想で近未来型のモバイルアプリケーションの企画と開発を行い,発想から納品までの実践的なソフトウェア開発手法を学ぶということである.
\par
未来大は,アプリケーションを2種類開発する.アプリケーションは「Cool Japanimation」と「Rhyth/Walk」である.作成するアプリケーションの対象プラットフォームは,Android,iOS,と両端末のWebブラウザであるHTML5を用いる.
\par
本プロジェクトの最終ゴールである企業報告に向けてアプリケーション案を考案し,ドキュメントを作成し,実際にプログラミングを行いアプリケーションを実装し,活動する.
\par
ミライケータイプロジェクトは,未来大,神奈工,長崎大の3大学合同で行われる.そこで,未来大が中心となって活動することが,未来大の目標である.
\par
また,ミライケータイプロジェクトでは,各大学の特徴を活かし,活動するということで,未来大は,ソフトウェア開発に関する活動に重きをおき,ドキュメントの作成やグループ活動,プログラミングについて重点的に力を入れ,活動をする.また,今年度は新たな取り組みとして未来大が中心となって,ビジネスモデルを企画する.
\par
また新たな取り組みとして,iOSの開発に今年6月に発表された新言語であるSwiftを導入することにより,実験的にアプリケーションの開発を行う.このように新言語にも挑戦していくことも未来大の課題のひとつである.
\par
3大学合同で行うプロジェクト活動を通じて,アプリケーションを提案する技術を習得するだけではなく,大人数で活動することや,遠隔地での活動による意見や意識の共有の難しさ,達成したときの喜び,グループで活動することでの楽しさや大変さなどを学び,各作業を終えるたびに人間的に成長するなどして,全員で1つの目標に向かって努力し続ける精神を養いたい.
\par
また同時に,ミライケータイプロジェクトの目標は,作成したアプリケーションに関するドキュメントの仕様をすべて,アプリケーションに実装することであるので,ミライケータイプロジェクトメンバ全員は,これを目標として,活動していく.
\bunseki{紺井 和人(未来大)}
