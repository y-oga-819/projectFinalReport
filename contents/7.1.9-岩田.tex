\subsection{岩田 一希}
\par
本プロジェクトにおいて,私はAndroid版の「Cool Japanimation」の開発を担当した.さらに,未来大のAndroid班のリーダを務めた.
前期のAndroid班では,参考書のサンプルアプリを解読しながら,技術習得を行っていた.
前期の中間発表会では,全体スライドを作成するなど,ドキュメントの作成も行っていた.
後期は,主に「Cool Japanimation」の開発,マージの担当を担っていた.

4月
\begin{itemize}
\item MLリーダに就任
\item 未来大用MLを作成
\item 本プロジェクト全体用MLを作成
\end{itemize}
5月
\begin{itemize}
\item 本プロジェクト全体用MLを作成
\item Android技術習得班に所属
\item Android技術習得班リーダに就任
\item Android技術習得のスケジュール作成
\item Android技術習得班の進捗を管理
\item 合宿用のAndroid技術習得のデモ作成
\item Dropbox管理リーダに就任
\item 本プロジェクト用のGoogleアカウント作成
\item 未来大用のDropboxの共有フォルダ作成
\item アプリケーションのアイディアのブラッシュアップ
\item アプリケーションのアイディアをアイディア提案シートに記入
\item アプリケーションのアイディアの類似アプリケーション調査
\item アプリケーションのアイディアの発表用スライドの作成
\end{itemize}
6月
\begin{itemize}
\item Android技術習得班の進捗を管理
\item アプリケーションのアイディアの発表用スライドのブラッシュアップ
\item 第一回合同合宿に参加
\item 第一回合同合宿1日目にアプリケーションのアイディアを発表
\item 第一回合同合宿で大学混合チームのリーダに就任
\item 第一回合同合宿2日目にアプリケーションのアイディアを発表
\item 消去されたDropboxのデータを復旧
\item MLの配信エラー状態を確認
\item Android開発班全体リーダに就任
\item Android開発班の進捗を管理
\item GoogleDrive管理リーダに就任
\item GoogleDriveに共有用フォルダを作成
\item 「Cool Japanimation」の要求定義書を作成
\item 「Cool Japanimation」の要件定義書を作成
\end{itemize}
7月
\begin{itemize}
\item 「Cool Japanimation」のサービス仕様書を作成
\item 中間報告会の発表用スライドを作成
\item 中間報告会の「Cool Japanimation」のスライドを作成
\item 中間報告会の発表用スライドをブラッシュアップ
\item 中間報告会の「Cool Japanimation」のスライドをブラッシュアップ
\item 「Cool Japanimation」の詳細仕様書を作成
\item 個人報告書を作成
\item グループ報告書の表紙,概要,Abstract,目次,4.1.2,個人作業を作成
\item Android開発用の技術習得
\item 中間報告会で発表
\end{itemize}
8月
\begin{itemize}
\item アプリケーション開発
\item 必要な技術の習得
\end{itemize}
9月
\begin{itemize}
\item アプリケーションの開発
\item 必要な技術の習得
\end{itemize}
10月
\begin{itemize}
\item アプリケーション開発
\item アカデミックリンクの準備
\item Android班でgithubを導入
\item キャンパスベンチャーグランプリ準備
\end{itemize}
11月
\begin{itemize}
\item 第二回合宿に参加
\item アプリケーションの開発
\item 成果発表会の準備
\end{itemize}
12月
\begin{itemize}
\item アプリケーションの開発
\item 成果発表会デモンストレーション作成
\item 成果発表会に参加
\item 最終報告書の作成
\item 企業報告会準備
\item 秋葉原での課外成果発表会の準備
\item 企業報告会・秋葉原での課外成果発表会での,「Cool Japanimation」のデモンストレーション担当に就任
\end{itemize}
1月
\begin{itemize}
\item 「Cool Japanimation」の仕様書の修正
\item アプリケーションのテストを実施
\item 「Cool Japanimation」のデモンストレーションを作成
\item 最終報告書の作成
\item 企業報告会準備
\item 秋葉原での課外成果発表会の準備
\end{itemize}
2月
\begin{itemize}
\item 秋葉原での課外成果発表会に参加
\end{itemize}

\par
特に頑張ったこと
\par
\par
本プロジェクトにおいて,私は前期,後期通してAndroid班に所属し,開発だけではなく,リーダとして,進捗管理などを担当した.
前期の最初の内は,情報共有の手段であるで,「ML」や,「DropBox」「GoogleDrive」の管理担当として,登録や,データ復旧などを担当していた.
さらに,開発アプリが決定した後は,「Cool Japanimation」を主に担当し,仕様書や,ビジネスモデルの考案に携わった.仕様書を考える段階においては,何度も長崎大学と会議をした.
夏休み前にはおおまかな仕様が決まったが,実際に開発を始めると,詰まってない仕様において,AndroidとHTML5で相違が発生してしまって,示し合わせをするのが大変だった.
さらにAndroid班は,「Cool Japanimation」「Rhyth/Walk」どちらの開発にも全員で携わろうとしたため,進捗管理や,それぞれのタスク管理などが大変になってしまった.
また,Android「Cool Japanimation」の開発において,私はソースコードのマージを担当した.最初はマージに苦労しないようにgithubを活用しようとしたが,使い方を間違えてしまい,手動でマージすることになった.
現在は,githubで管理できるような環境を整えることが出来たので,必要なときにマージすることが可能となっている.
\par
今後としては,2月の企業報告会,秋葉原での課外成果発表会に向けて,アプリの魅力が伝わるような,デモンストレーション作り,ビジネスモデル考案に力を入れたいと考えている.
さらに,アプリケーションの開発に関して,まだ,魅力が伝わるデモンストレーションができる開発状況ではないので,早急に,アプリケーションを作成する必要があるので,そちらにも力を入れて頑張りたいと思う.
\bunseki{岩田 一希(未来大)}
