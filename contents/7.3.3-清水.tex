\subsection{清水 彰人}
6 月
\begin{itemize}
\item Cool Japanimation サーバ班として新規参加
\end{itemize}
7 月
\begin{itemize}
\item 中間報告書の担当部分の作成
\item サーバにおける技術習得
\item オープンキャンパスの用意
\end{itemize}
8 月
\begin{itemize}
\item サーバにおける技術習得
\end{itemize}
9 月
\begin{itemize}
\item アプリケーションの開発
\item 必要な技術の習得
\end{itemize}
10 月
\begin{itemize}
\item アプリケーションの開発
\item チャットプログラムの選別と選択
\end{itemize}
11 月
\begin{itemize}
\item アプリケーション完成
\item チャット機能に関してのアプリ班との連携
\item 第二回合同合宿
\end{itemize}
12 月
\begin{itemize}
\item チャット機能の修正
\item プロジェクト最終報告
\end{itemize}
\par
活動内容
  私はサーバ班として6月からプロジェクトに参加した.チャット機能を担当し,サーバの中にチャット用のサーバを立てた.
  最初は突然の参加になったこともあり,まずは現状がどうなっているかということを把握することから始まった.このアプリケーションの構想
  などに関してはあまり関われてなかった部分も多々あったが,なかなか面白いアプリケーションだと思った.開発に携わっていくにあたって
  かなり実装が難しいと感じるところが多かった.チャット機能に関しては,もともと何種類かのプログラムを考えていた.
  しかし,それらのプログラムに関してはポートの問題であったり,様々な問題があったため実装できなかった.そのために当初の
  予定よりもチャット機能の実装が遅くなってしまったという点で、期限を守りながら開発を進めるということの難しさを感じた.
  これから様々なプログラムを組んでいく中で,スケジュールを考えながらの開発を確実にできるようにしていくように活かしていき,企業報告会までに
  現在のチャット機能のUIなどを改良していく予定である.
  
  
\bunseki{清水 彰人(長崎大)}
