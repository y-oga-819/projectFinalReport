\subsection{神奈川工科大学}
\par 神奈工は企画から開発までの工程や,他大学との連携プロジェクトの学習を目標に前期のプロジェクト活動を行った.この活動によって神奈工は企画立案力の向上,技術や知識の向上,そしてリーダシップ,メンバシップの向上の3つの成果を上げることができた.
\par 私たちはアプリケーションの開発について企画から携わることができた.企画はまず神奈工内でブレインストーミングを行い,KJ法によってアイディアをまとめていった.それをさらに各大学でプレゼンテーション,および評価を行って最終的に2つのアイディアに数を絞った.さらに各プロセスではレビューをして各アイディアに反映させていった.企画は普段私たちが行うことはほとんどなく,手探りの状況が多かった.現在では企画において何をすればよいかがイメージできるようになり,戸惑うことは少なくなった.
\par 技術や知識の習得は先代の先輩から勉強会を行う形式で中心に行った.第一回合同合宿に向けたデモアプリケーションの作成では必要な技術要素を習得するために自分たちで勉強会を行う場面もあった.これらの習得は特定の技術要素にこだわらず,幅広く行うことができた.またメンバの進行を合わせて行われているため,差の少ない習得が行えた.今後はGitHubなど開発面での他大学との連携も検討する予定である.
\par 本プロジェクトは他大学との連携形態をとっている.情報の共有,連絡手段にはWikiやメーリングリスト,Skype,GoogleDriveを用いた.連絡の最中に認識のずれを見ることがあったが,早急にすべての関係者で意見を取り合ってまとめることができた.こうしたことからWikiの情報や各ドキュメントによって認識の共有を行うことの重要性を学ぶことができた.また例年よりも少人数という環境も持ち合わせており,作業量の調整に苦慮したが,メンバがいまできることを見極めてタスクの振り分け,フォローをすることができた.
※\bunseki{遠藤 崇(神奈工)}
