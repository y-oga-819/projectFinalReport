\subsection{Swiftでの開発}
\par
ミライケータイプロジェクトでは, iOSでのアプリケーション開発は例年Objective-Cという言語で実装されていた.我々も開発当初は同じようにその言語で開発を行っていたが,2014年6月にアップル社がSwiftという言語を発表した.このことから,ミライと冠するプロジェクトであるならより先進的なものを使うべきだと思い,新しい言語SwiftでのiOSアプリケーション開発を行うことを決めた.開発を始めた当初は,発表直後ということもあり,Rhyth/Walkの開発で必要な音楽関係を扱うリファレンスが少なく苦労することも多かった.また,バージョンアップの頻度が激しく,たびたび仕様が変わるため,バージョンが変わるたびにそれに対応しなければならなかった.ただ,Objective-Cを学んでいたため,応用できる部分も少なからずあり,それは大いに助かった.
\bunseki{三栖 惇(未来大)}
